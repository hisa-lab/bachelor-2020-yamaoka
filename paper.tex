\documentclass[11pt]{sty/oecu-thesis}
\usepackage{cite}
\usepackage[dvipdfmx]{graphicx}
\usepackage{subfigure}
\usepackage{sty/lcaption}
\usepackage{times}
\usepackage{url}
\usepackage{amsmath}

% もう少し具体的なタイトルにする。
\title[スクリプト言語のパフォーマンスについての検証]{スクリプト言語のパフォーマンスについての検証}
\author{山岡 風太}
\date{{令和}\rensuji{2}年\rensuji{12}月\rensuji{14}日}
\学生番号{HT18A102}
\指導教員{久松 潤之 准教授}

% 特別研究の場合はコメントをはずす.卒業研究の場合はコメントアウトする.
%\論文種別{特別研究論文}
\年度{令和2}

\所属{総合情報学部 情報学科}


\begin{document}


\newcommand{\insertfigeps}[2]{%
  \begin{figure}[tb]
    \begin{center}
      \leavevmode
      \includegraphics[width=.80\textwidth]%
      {figure/#1.eps}
      \lcaption{#2}
      \label{fig:#1}
    \end{center}
  \end{figure}}

\newcommand{\insertfigpng}[2]{%
  \begin{figure}[tb]
    \begin{center}
      \leavevmode
      \includegraphics[width=.80\textwidth]%
      {figure/#1.png}
      \lcaption{#2}
      \label{fig:#1}
    \end{center}
  \end{figure}}



%\makeextratitle
\maketitle
\pagenumbering{roman}
\begin{abstract}
  近年、スマートフォン等の個人用端末の普及によってインターネット上のWEBサービスは多くの人に利用されている。 
  Web サービスを構築する際には PHP 等のスクリプト言語と呼ばれるプログラミング言語を用いるのが一般的である。
  スクリプト言語はC++ 等と比較するとコンパイルが不要という利点があるが、それによって実行時間が長くなりがちと言う問題が一般的に知られている。
  しかし、近年、需要の増大にともなって、各種スクリプト言語はバージョンアップ時に実行時間を含めたパフォーマンス問題の改善に力を入れている。
  また、Node.js など近年出現したスクリプト言語あるいはフレームワークは、初期段階からパフォーマンス問題に対して意欲的に取り組んでいる。
  そのため、これらのスクリプト言語のパフォーマンスは、従来考えられているよりも相当程度に改善されている可能性がある。
  そこで、本研究では各種スクリプト言語の最新のバージョンを対象にして、実行時間のパフォーマンスに関する検証を行う。
  
\keywords % 主な用語
  スクリプト言語\quad
  Ubuntu\quad
  Python~\cite{Python}\quad
  PHP\quad
  Ruby\quad  
  Node.js\quad

\end{abstract}



\tableofcontents
% 以下の二つは,論文のフォーマットにそっていないが,
% 確認用のためにつける.論文提出時には,コメントアウトする.
\listoffigures
\listoftables
\cleardoublepage

\setcounter{page}{1}
\pagenumbering{arabic}

\chapter{はじめに}
\label{cha:intro}
近年、Webサービスはスクリプト言語を使用することが多い。また、スクリプト言語はwebサービスの需要増加に伴って増加傾向にある。
スクリプト言語はC++などの言語に比べて実行速度が遅かったが、近年スクリプト言語開発者による最適化が進んでいる。
それによって、スクリプト言語のパフォーマンスは従来考えられていたものより改善されている可能性がある。
本研究では、4つのスクリプト言語を用意してパフォーマンスの検証を行った。
本論文の構成は以下の通りである。
まず、\ref{cha:related} 章では、本研究で参考にしたwebサイトについて記す。
その後、\ref{cha:script-language}章で計測に使用するスクリプト言語4種類について概要を述べる。
また、\ref{cha:environment}章では計測に使ったpcの構成とスクリプト言語のバージョンについて記す。
\ref{cha:program}章は計測に使用したプログラムの概要を述べる。
\ref{cha:result}章では計測結果を表にして、その結果を考察する。
最後に、\ref{cha:conclusion} 章では、本論文のまとめと今後の課題を述べる。

\chapter{関連研究}
\label{cha:related}
引用の書き方

例えば、\cite{NS2} では、~~


\chapter{計測に使用したスクリプト言語の概要}
\label{cha:script-language}
計測には下記のスクリプト言語を使用した。これらの言語は現在、人気のあるスクリプト言語でweb開発によく使われている。
\section{Python3}
pythonは1980年代に考案されたスクリプト言語で、本研究では2008年にリリースされたpython3を用いた。
同じ処理を行うプログラムは誰が書いても同じになるように開発された言語で、
例えば、インデントによってプログラムのブロックを定義することができるという特徴がある。
ほとんどのプログラミング言語ではブロックを定義する場合にインデントは意味を持たず、
プログラムの可読性の向上のために任意で行われていたものだが、
この特徴によりプログラムの構造が似たものとなるため、プログラムの執筆者でなくとも可読性が高い。
統計や解析、分析に長けたライブラリが充実しており、研究分野でよく使われる。

\section{PHP}
PHPは1995に考案された動的なWebページを生成するツールが起源のスクリプト言語である。
Webアプリケーション開発に関する標準ライブラリが豊富で、PHPは様々なwebサーバーで使用されている。
PHPの制御系はTHMLなどに埋め込まれた"<?php ?>"で囲われた部分を読み取って解釈しプログラムを実行する。
また、cやperlに強く影響を受けており、文法やプログラムの構造に類似している点が多く、それらのプログラミング経験者は学習が容易である。

\section{node.js}
node.jsは非同期の処理を行うアプリケーションを作成するために2011年ごろ作られたスクリプト言語の1つでサーバーサイドのJavaScript環境である。
メモリ消費量が少ないため、小規模の運用を行う場合では、他の環境と比べると総合的なパフォーマンスが高い。
通信やファイルの読み書きを、処理待ちによってブロックされることが少ないノンブロックI/O方式が使用されている。
従来のサーバーではC10K問題というサーバーへの接続台数が一万台を超えると、処理が遅くなるという問題があるが、
ノンブロッキングI/Oによってnode.jsを使うだけで意識せずとも解決される。
このような機能によって、大量のアクセスに対応できるのでリアルタイム性が問われるウェブサイトの作成に長けている。

\section{ruby}
rubyは日本で1995年に開発されたプログラミング言語で、日本発の言語では初めて国際標準規格に認定された。 
また、日本発の言語であるため、日本語の資料が豊富で学習が容易である。
また、プログラムの記述量が少ない。
例えば文字を出力したい場合python3ではprint ("hello world")とする所を
rubyではp "hello world"という風に短く記述することができる。
また、Ruby on Railsと言われるwebアプリケーション開発用のフレームワークが有名であり、これを使ったwebサイトが多く存在している。


\chapter{計測環境}
\label{cha:environment}
本研究で使用したpcの構成は下記のものを使用した。
\begin{itemize}
  \item microsoft windows 10 home
  \item CPU Intel(R) Core(TM) i5-8400 CPU @ 2.80GHz
  \item memory 16GB
\end{itemize}

windows上でwsl2を使い、linuxの仮想環境を構築して計測を行った。\\
linuxの操作には"ubuntu 20.04.1"を使用した。

プログラミング言語のバージョンは下記のものを使用した。
\begin{itemize}
  \item python 3.8.2
  \item PHP 7.4.9
 \item node.js 12.18.3
  \item ruby 2.7.1
\end{itemize}
これらのプログラミング言語のバージョンは2020/8/29時点の最新バージョンである。

\chapter{計測したプログラム}
\label{cha:program}

\section{フィボナッチ数列}
スクリプト言語の再起呼び出しの性能を測定するために、フィボナッチ数列を算出するプログラムを作成した。
フィボナッチ数列とは「1,1,2,3,5,8,13,21,34,55,89,144」という様に数列のそれぞれの数が1つ前と2つ前の数の和になっている。
フィボナッチ数列は漸化式\ref{eq:fibonacci} と表せる。
\begin{eqnarray} \label{eq:fibonacci}
  F_{0}&=&0 \nonumber \\
  F_{1}&=&1 \\
  F_{n+1}&=&F_{n}+F_{n+1}(n≥0)\nonumber
\end{eqnarray}
この漸化式を再起処理を使って再現し実装する。
Python3で実装したものを図\ref{fig:f-py}に示す。

\section{円周率の算出}
スクリプト言語の計算能力を測定するために、円周率を求めるプログラムを作成した。円周率を求めるためにライプニッツ級数を使用した。
ライプニッツ級数は次の級数\ref{eq:leibniz}と表せる。
\begin{eqnarray} \label{eq:leibniz}
\sum_{n=0}^{\infty}\frac{(-1)^n}{2n+1}=\frac{\pi}{4}
\end{eqnarray}
級数の実行回数を増やすと円周率の精度が上がる。
導出法の中では収束が遅く効率は悪いが、現在のコンピューターではほかの方法を使うと非常に早く収束してしまい測定が難しい。
よって、ライプニッツ級数はプログラミング言語の性能を測るのに適していると考えた。
Python3で実装したものを図\ref{fig:p-py}に示す。

\section{正規表現}
スクリプト言語の正規表現に関する処理能力を測定するためにURLを判別するプログラムを作成した。
任意の行数のurlが混ざったテキストファイルを開き、そのデータから正規表現を使ってURLを取り出し出力する。
URLの判別には正規表現を使う。行の数を増やしていき、実行の速度を測る。
URLの判別には下記の正規表現を使った。Python3で実装したものを図\ref{fig:s-py}に示す。

\insertfigpng{f-py}{Python3 フィボナッチ数列}
\insertfigpng{p-py}{Python3 円周率の算出}
\insertfigpng{s-py}{Python3 正規表現}
\chapter{計測結果}
\label{cha:result}
計測結果を書く
\section{フィボナッチ数列}
n番目フィボナッチ数を求めるプログラムの実行速度をそれぞれ20回ずつ測定した。
計20回の結果の平均は図\ref{fig:f-average}となった。
また、計20回の結果の分散は図\ref{fig:f-dispersion}となった。
どのスクリプト言語も導出するフィボナッチ数が大きくなるほど実行時間が伸びる傾向にある。

\insertfigpng{f-average}{フィボナッチ数列 平均}
図\ref{fig:f-dispersion}
\insertfigpng{f-dispersion}{フィボナッチ数列 分散}

\section{円周率の算出}
図\ref{fig:p-average}
\insertfigpng{p-average}{円周率の算出 平均}
図\ref{fig:p-dispersion}
\insertfigpng{p-dispersion}{円周率の算出 分散}


\section{正規表現}
図\ref{fig:s-average}
\insertfigpng{s-average}{正規表現 平均}
図\ref{fig:s-dispersion}
\insertfigpng{s-dispersion}{正規表現 分散}

\chapter{まとめと今後の課題}
\label{cha:conclusion}

 本研究では、主要な 4 種類のスクリプト言語である Python, PHP, Ruby, Node.js のパフォーマンスの検証を実行速度の観点から行った。
 また、その結果を表にして示しその考察を行った。

 今後の課題としては、パフォーマンス検証の評価を実行速度だけではなく、CPUの使用率やメモリー使用率などを加味して行うべきだろう。
\acknowledgment
本研究と本論文を終えるにあたり、御指導、御教授を頂いた久松潤之准教授に
深く感謝致します。また、学生生活を通じて、基礎的な学問、学問に取り組む
姿勢を御教授頂いた、登尾啓史教授、升谷保博教
授、渡邊郁教授、南角茂樹教授、鴻巣敏之教授、北嶋暁教授、大西克彦
教授に深く感謝致します。

本研究期間中、本研究に対する貴重な御意見、御協力を頂きました久松研究室
の皆様に心から御礼申し上げます。


\bibliographystyle{junsrt}
\bibliography{bib/myrefs}


\end{document}

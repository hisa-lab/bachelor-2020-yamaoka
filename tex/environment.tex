\chapter{計測環境}
\label{cha:environment}
\section{コンピューターの構成}
本研究で使用したコンピューターの構成は下記のものを使用した。
\begin{itemize}
  \item microsoft windows 10 home
  \item CPU Intel(R) Core(TM) i5-8400 CPU @ 2.80GHz
  \item memory 16GB
\end{itemize}

\section{Linux仮想環境}
windows上でWLS2を使用し、Linuxの仮想環境を構築して計測を行った。
WSL2とはwindows上で仮想マシンを使い、Linuxカーネルが動作する仕組みのことである。
また、Linuxの操作にはUbuntu\cite{Ubuntu}を使用した。UbuntuはデスクトップLinuxディストリビューションの一つで、Linuxの操作に使用する。
バージョンは"Ubuntu 20.04.1"を使用した。

\section{プログラミング言語のバージョン}
プログラミング言語のバージョンは下記のものを使用した。
\begin{itemize}
  \item Python 3.8.2
  \item PHP 7.4.9
 \item node.js 12.18.3
  \item ruby 2.7.1
\end{itemize}
これらのプログラミング言語のバージョンは2020/8/29時点の最新バージョンである。
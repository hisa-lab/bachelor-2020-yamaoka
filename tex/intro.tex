\chapter{はじめに}
\label{cha:intro}
近年、Webサービスではスクリプト言語と呼ばれるプログラミング言語を用いるのが一般的である。また、スクリプト言語はwebサービスの需要増加に伴って増加傾向にある。
スクリプト言語はC++などの言語に比べて実行速度が遅かったが、近年ではスクリプト言語開発者による最適化が盛んに行われている。
それによって、スクリプト言語は従来考えられているよりも相当程度に改善されている可能性がある。
そこで、本研究ではPython3,PHP,node.js,rubyを対象にして、実行時間に関する検証を行う。\\
 本論文の構成は以下の通りである。\\
まず、\ref{cha:related} 章では、本研究で参考にしたwebサイトについて述べる。
その後、\ref{cha:script-language}章で計測に使用するスクリプト言語4種類について概要を述べる。
また、\ref{cha:environment}章では計測に使ったpcの構成とスクリプト言語のバージョンについて述べる。
\ref{cha:program}章は計測に使用したプログラムの概要を述べる。
\ref{cha:result}章ではプログラムの計測結果を表にし、その結果を考察する。
最後に、\ref{cha:conclusion} 章では、本論文のまとめと今後の課題を述べる。
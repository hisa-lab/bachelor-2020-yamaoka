\begin{abstract}
  近年、スマートフォン等の個人用端末の普及によってインターネット上のWEBサービスは多くの人に利用されている。 
  Web サービスを構築する際には PHP 等のスクリプト言語と呼ばれるプログラミング言語を用いるのが一般的である。
  スクリプト言語はC++ 等と比較するとコンパイルが不要という利点があるが、それによって実行時間が長くなりがちと言う問題が一般的に知られている。
  しかし、近年、需要の増大にともなって、各種スクリプト言語はバージョンアップ時に実行時間を含めたパフォーマンス問題の改善に力を入れている。
  また、node.js など近年出現したスクリプト言語あるいはフレームワークは、初期段階からパフォーマンス問題に対して意欲的に取り組んでいる。
  そのため、これらのスクリプト言語のパフォーマンスは、従来考えられているよりも相当程度に改善されている可能性がある。
  そこで、本研究では各種スクリプト言語の最新のバージョンを対象にして、実行時間のパフォーマンスに関する検証を行う。
  
\keywords % 主な用語
  スクリプト言語\quad
  Ubuntu\quad
  Python3\quad
  PHP\quad
  node.js\quad
  ruby\quad
\end{abstract}


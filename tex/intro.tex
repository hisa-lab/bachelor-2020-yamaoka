\chapter{はじめに}
\label{cha:intro}

Web サービスにおいては、Perl~\cite{Perl}, Ruby~\cite{Ruby}, PHP~\cite{PHP} を始めとした
スクリプト言語と呼ばれるプログラミング言語を用いて開発するのが一般的となっている。
これは、これらのプログラミング言語が C や C++ などに比べて、文字列操作や正規表現などの
Web サービスの開発に必要となる機能を豊富にサポートしている事や、Web サービスの開発に特化した
ライブラリやフレームワークが数多く提供されている事などに起因する。
スクリプト言語は Web 2.0~\cite{Web20} のキーワードの下、多くの人々や企業が Web サービスを
提供するようになるに伴って、急速に普及した。
さらに iPhone や Android などのスマートフォンと呼ばれる携帯端末が普及すると、Web ブラウザ上に
表示される Web ページがアプリケーションのように振る舞う、Web アプリケーションと呼ばれるものが
注目を集めるようになった。これは、異なる OS に対してアプリケーションを提供する際に、
Web ブラウザを利用する事によりプログラムを単一に保ち、開発コストを削減する効果が期待できる事などが
理由に挙げられる。
Web アプリケーションの開発に注目が集まるようになると、Go~\cite{Go} や Web ブラウザ上での動作を
前提としていた JavaScript をそれ以外の場面でも実行できるようにする Node.js~\cite{NodeJS} など
新しいプログラミング言語やそれに関連する技術も多数登場した。 

一般的に、スクリプト言語は C や C++ などのプログラミング言語に比べて実行速度が遅い。
しかし、前述したように Web サービスや Web アプリケーションの開発が盛んになるにつれて、
スクリプト言語開発者による最適化も盛んに行われるようになった。
例えば、PHP は 2015 年に新バージョンとなる PHP7~\cite{PHP7} がリリースされたが、
PHP7 は以前のバージョンである PHP5.6 と比較すると、ほとんどの互換性を維持したまま
約二倍の性能向上に成功している。さらに命令呼び出し回数の削減や検索手法の改善、メモリ使用量の削減なども
行われ、総合的なパフォーマンスが向上している。
このような性能向上を伴うアップデートはほとんどのスクリプト言語においても同様に行われており、
スクリプト言語の性能は従来考えられているよりも相当程度に改善されている可能性がある。

そこで本研究では、Web サービスや Web アプリケーションに用いられるスクリプト言語の
最新バージョンを対象に、改めて各種パフォーマンスに関する検証を行う。
具体的には、現在 Web サービスの開発に用いられる主要スクリプト言語である
Python~\cite{Python}, PHP, Node.js, Ruby の最新バージョンを用いて、
実行時間に関する比較検討を行う。

本論文の構成は以下の通りである。
\ref{cha:script-language}~章では、実行時間の比較検討の対象となる 4 種類のスクリプト言語の概要を述べる。
次に、\ref{cha:program}~章では、実行時間を比較検討するために本研究で作成したプログラムの概要を述べる。
さらに、\ref{cha:environment}~章では実行時間の比較検討に使用した計算機端末の構成と各種スクリプト言語の
バージョンについて説明し、\ref{cha:result}~章では比較検討結果について述べる。
最後に、\ref{cha:conclusion}~章では、本論文のまとめと今後の課題を述べる。
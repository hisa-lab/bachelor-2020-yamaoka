\chapter{計測したプログラム}
\label{cha:program}

\section{フィボナッチ数列}
スクリプト言語の再起呼び出しの性能を測定するために、フィボナッチ数列を算出するプログラムを作成した。
フィボナッチ数列とは「1,1,2,3,5,8,13,21,34,55,89,144」という様に数列のそれぞれの数が1つ前と2つ前の数の和になっている。
フィボナッチ数列は漸化式\ref{eq:fibonacci} と表せる。
\begin{eqnarray} \label{eq:fibonacci}
  F_{0}&=&0 \nonumber \\
  F_{1}&=&1 \\
  F_{n+1}&=&F_{n}+F_{n+1}(n≥0)\nonumber
\end{eqnarray}
この漸化式を再起処理を使って再現し実装する。
Python3で実装したものを図\ref{fig:f-py}に示す。

\section{円周率の算出}
スクリプト言語の計算能力を測定するために、円周率を求めるプログラムを作成した。円周率を求めるためにライプニッツ級数を使用した。
ライプニッツ級数は次の級数\ref{eq:leibniz}と表せる。
\begin{eqnarray} \label{eq:leibniz}
\sum_{n=0}^{\infty}\frac{(-1)^n}{2n+1}=\frac{\pi}{4}
\end{eqnarray}
級数の実行回数を増やすと円周率の精度が上がる。
導出法の中では収束が遅く効率は悪いが、現在のコンピューターではほかの方法を使うと非常に早く収束してしまい測定が難しい。
よって、ライプニッツ級数はプログラミング言語の性能を測るのに適していると考えた。
Python3で実装したものを図\ref{fig:p-py}に示す。

\section{正規表現}
スクリプト言語の正規表現の処理能力を測定するためにURLを判別するプログラムを作成した。
任意の行数のurlが混ざったテキストファイルを開き、そのデータから正規表現を使ってURLを取り出し出力する。
URLの判別には正規表現を使う。行の数を増やしていき、実行の速度を測る。
URLの判別には下記の正規表現を使った。Python3で実装したものを図\ref{fig:s-py}に示す。

\insertfigpng{f-py}{Python3 フィボナッチ数列}
\insertfigpng{p-py}{Python3 円周率の算出}
\insertfigpng{s-py}{Python3 正規表現}
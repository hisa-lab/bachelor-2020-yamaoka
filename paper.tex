\documentclass[11pt]{sty/oecu-thesis}
\usepackage{cite}
\usepackage[dvipdfmx]{graphicx}
\usepackage{subfigure}
\usepackage{sty/lcaption}
\usepackage{times}
\usepackage{url}
\usepackage{amsmath}

% もう少し具体的なタイトルにする。
\title[タイトルタイトル]{タイトル\\タイトル}
\author{名前 名前}
\date{{平成}\rensuji{99}年\rensuji{99}月\rensuji{99}日}
\学生番号{HT99A999}
\指導教員{久松 潤之 准教授}

% 特別研究の場合はコメントをはずす.卒業研究の場合はコメントアウトする.
%\論文種別{特別研究論文}
\年度{平成99}

\所属{総合情報学部 情報学科}


\begin{document}


\newcommand{\insertfigeps}[2]{%
  \begin{figure}[tb]
    \begin{center}
      \leavevmode
      \includegraphics[width=.80\textwidth]%
      {figure/#1.eps}
      \lcaption{#2}
      \label{fig:#1}
    \end{center}
  \end{figure}}

\newcommand{\insertfigpng}[2]{%
  \begin{figure}[tb]
    \begin{center}
      \leavevmode
      \includegraphics[width=.80\textwidth]%
      {figure/#1.png}
      \lcaption{#2}
      \label{fig:#1}
    \end{center}
  \end{figure}}



%\makeextratitle
\maketitle
\pagenumbering{roman}
\begin{abstract}
  近年、スマートフォン等の個人用端末の普及によってインターネット上のWEBサービスは多くの人に利用されている。 
  Web サービスを構築する際には PHP 等のスクリプト言語と呼ばれるプログラミング言語を用いるのが一般的である。
  スクリプト言語はC++ 等と比較するとコンパイルが不要という利点があるが、それによって実行時間が長くなりがちと言う問題が一般的に知られている。
  しかし、近年、需要の増大にともなって、各種スクリプト言語はバージョンアップ時に実行時間を含めたパフォーマンス問題の改善に力を入れている。
  また、Node.js など近年出現したスクリプト言語あるいはフレームワークは、初期段階からパフォーマンス問題に対して意欲的に取り組んでいる。
  そのため、これらのスクリプト言語のパフォーマンスは、従来考えられているよりも相当程度に改善されている可能性がある。
  そこで、本研究では各種スクリプト言語の最新のバージョンを対象にして、実行時間のパフォーマンスに関する検証を行う。
  
\keywords % 主な用語
  スクリプト言語\quad
  Ubuntu\quad
  Python~\cite{Python}\quad
  PHP\quad
  Ruby\quad  
  Node.js\quad

\end{abstract}



\tableofcontents
% 以下の二つは,論文のフォーマットにそっていないが,
% 確認用のためにつける.論文提出時には,コメントアウトする.
\listoffigures
\listoftables
\cleardoublepage

\setcounter{page}{1}
\pagenumbering{arabic}

\chapter{はじめに}
\label{cha:intro}
近年、Webサービスはスクリプト言語を使用することが多い。また、スクリプト言語はwebサービスの需要増加に伴って増加傾向にある。
スクリプト言語はC++などの言語に比べて実行速度が遅かったが、近年スクリプト言語開発者による最適化が進んでいる。
それによって、スクリプト言語のパフォーマンスは従来考えられていたものより改善されている可能性がある。
本研究では、4つのスクリプト言語を用意してパフォーマンスの検証を行った。
本論文の構成は以下の通りである。
まず、\ref{cha:related} 章では、本研究で参考にしたwebサイトについて記す。
その後、\ref{cha:script-language}章で計測に使用するスクリプト言語4種類について概要を述べる。
また、\ref{cha:environment}章では計測に使ったpcの構成とスクリプト言語のバージョンについて記す。
\ref{cha:program}章は計測に使用したプログラムの概要を述べる。
\ref{cha:result}章では計測結果を表にして、その結果を考察する。
最後に、\ref{cha:conclusion} 章では、本論文のまとめと今後の課題を述べる。
\chapter{関連研究}
\label{cha:related}
引用の書き方

例えば、\cite{NS2} では、~~

\chapter{XXX}
\label{cha:xxx}
このファイルは、ファイル名や章タイトル、そして、label を適宜書き換えること。

\insertfigeps{sample-eps}{eps 画像の貼り付けの例}
\insertfigpng{sample-png}{png 画像の貼り付けの例}
\chapter{まとめと今後の課題}
\label{cha:conclusion}

 本研究では、主要な 4 種類のスクリプト言語である Python, PHP, Ruby, Node.js のパフォーマンスの検証を実行速度の観点から行った。
 また、その結果を表にして示しその考察を行った。

 今後の課題としては、パフォーマンス検証の評価を実行速度だけではなく、CPUの使用率やメモリー使用率などを加味して行うべきだろう。
\acknowledgment
本研究と本論文を終えるにあたり、御指導、御教授を頂いた久松潤之准教授に
深く感謝致します。また、学生生活を通じて、基礎的な学問、学問に取り組む
姿勢を御教授頂いた、登尾啓史教授、升谷保博教
授、渡邊郁教授、南角茂樹教授、鴻巣敏之教授、北嶋暁教授、大西克彦
教授に深く感謝致します。

本研究期間中、本研究に対する貴重な御意見、御協力を頂きました久松研究室
の皆様に心から御礼申し上げます。


\bibliographystyle{junsrt}
\bibliography{bib/myrefs}


\end{document}

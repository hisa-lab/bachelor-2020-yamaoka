\chapter{計測に使用したスクリプト言語の概要}
\label{cha:script-language}
計測には下記のスクリプト言語を使用した。
\section{Python3}
pythonは1980年代に考案されたオープンソースのオブジェクト指向スクリプト言語である。
本研究では2008年にリリースされたバージョンの一つであるpython3を用いた。
同じ処理を行うプログラムは誰が書いても同じになるように開発された言語で、
例えば、インデントによってプログラムのブロックを定義することができるという特徴がある。
ほとんどのプログラミング言語ではブロックを定義する場合にインデントは意味を持たず、
プログラムの可読性の向上のために任意で行われていたものだが、
この特徴によりプログラムの構造が似たものとなるため、プログラムの執筆者でなくとも可読性が高い。
統計や解析、分析に長けたライブラリが充実しており、研究分野でよく使われる。

\section{PHP}
PHPは1995に考案されたオープンソースのスクリプト言語である。
PHPはHypertext Preprocessorの略で動的なWebページを生成する目的で作られた。
そのため、Webアプリケーション開発に関する標準ライブラリが豊富で、PHPは様々なwebサーバーで使用されている。
例えば、PHPで書かれたwebサービスにはFacebookやwikipedia、slackなどがある。
PHPの制御系は\verb|<?php ?>|で囲われた部分を読み取って解釈しプログラムを実行する。また、マークアップ型言語であるHTMLに埋め込まれて使用する場合も多い。
また、cやperlに強く影響を受けており、文法やプログラムの構造に類似している点が多く、それらのプログラミング経験者は学習が容易である。

\section{node.js}
node.jsは非同期の処理を行うアプリケーションを作成するために2011年ごろ作られたオープンソースのスクリプト言語の1つでサーバーサイドのJavaScript環境である。
メモリ消費量が少ないため、小規模の運用を行う場合では、他の環境と比べると総合的なパフォーマンスが高い。
通信やファイルの読み書きを、処理待ちによってブロックされることが少ないノンブロックI/O方式が使用されている。
従来のサーバーではC10K問題というサーバーへの接続台数が一万台を超えると、処理が遅くなるという問題があるが、
ノンブロッキングI/Oによってnode.jsを使うだけで意識せずとも解決される。
このような機能によって、大量のアクセスに対応できるのでリアルタイム性が問われるウェブサイトの作成に長けている。

\section{ruby}
rubyは日本で1995年に開発されたオープンソースのオブジェクト指向プログラミング言語で、日本発の言語では初めて国際標準規格に認定された。
Enjoy Programming!を設計思想として開発された言語で、プログラムの記述量が少ない、構文がシンプル、ライブラリが高機能などの特徴がある。
プログラムの記述量が少ないという特徴を示す例を挙げると、例えば"Hello World"と出力したい場合には、python3はprint ("Hello World")と書くことができるが、
rubyではp "Hello World"という風に非常に短く記述することができる。
また、Ruby on Railsと言われるwebアプリケーション開発用のフレームワークが有名であり、これを使ったwebサイトが多く存在している。
さらに、日本発の言語であるため、日本語の資料やウェブサイトが豊富で学習が容易である。
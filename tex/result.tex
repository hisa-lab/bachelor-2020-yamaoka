\chapter{計測結果}
\label{cha:result}
計測結果を書く
\section{フィボナッチ数列}
n番目フィボナッチ数を求めるプログラムの実行速度をそれぞれ20回ずつ測定した。
計20回の結果の平均は図\ref{fig:f-average}となった。
また、計20回の結果の分散は図\ref{fig:f-dispersion}となった。
どのスクリプト言語も導出するフィボナッチ数が大きくなるほど実行時間が伸びる傾向にある。

\insertfigpng{f-average}{フィボナッチ数列 平均}
図\ref{fig:f-dispersion}
\insertfigpng{f-dispersion}{フィボナッチ数列 分散}

\section{円周率の算出}
図\ref{fig:p-average}
\insertfigpng{p-average}{円周率の算出 平均}
図\ref{fig:p-dispersion}
\insertfigpng{p-dispersion}{円周率の算出 分散}


\section{正規表現}
図\ref{fig:s-average}
\insertfigpng{s-average}{正規表現 平均}
図\ref{fig:s-dispersion}
\insertfigpng{s-dispersion}{正規表現 分散}

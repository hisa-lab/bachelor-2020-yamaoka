\chapter{計測に使用したスクリプト言語の概要}
\label{cha:script-language}
計測には下記のスクリプト言語を使用した。
\begin{itemize}
  \item Python3
    \begin{itemize}
      \item インデントが構文規則として決められている
      \item 統計学 解析 分析に長けたライブラリが充実している
      \item シンプルで見やすいコードを自然に書くことができる
    \end{itemize}
  \item PHP
    \begin{itemize}
      \item PHPは動的なWebページを生成するツールが起源
      \item Webアプリケーション開発に関する標準ライブラリが豊富
      \item c++やjavaに強く影響を受けており それらの経験者は学習が容易
    \end{itemize}
 \item node.js
    \begin{itemize}
      \item JavaScript実行環境の1つ サーバーサイドJavaScript環境
      \item 大量のアクセスに対応できるのでリアルタイム性が問われるウェブサイトの作成に長けている
      \item 処理が軽量でメモリーの使用量が少ない
    \end{itemize}
  \item ruby
    \begin{itemize}
      \item 日本で開発されたプログラミング言語であるので 日本語の資料が豊富
      \item ウェブサイトを作成することに長けている
      \item プログラムの記述量が少ない
    \end{itemize}
\end{itemize}
これらの言語は現在、人気のあるスクリプト言語でweb開発によく使われている。よって上記のスクリプト言語を本研究の研究対象に選んだ。

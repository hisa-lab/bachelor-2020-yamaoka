\chapter{はじめに}
\label{cha:intro}
近年、Webサービスではスクリプト言語と呼ばれるプログラミング言語を用いるのが一般的である。また、スクリプト言語はwebサービスの需要増加に伴って増加傾向にある。
スクリプト言語はC++などの言語に比べて実行速度が遅かったが、近年ではスクリプト言語開発者による最適化が盛んに行われている。
例えばPHPでは2015年にメジャーパッチPHP7\cite{PHPchange}がリリースされた。
PHP7は以前のバージョンであるPHP5.6と比較すると、ほとんどの互換性を維持したまま約二倍の性能向上に成功している。
さらに命令呼び出し回数の削減や検索手法の改善、メモリ使用量の削減なども行われ、総合的なパフォーマンスが向上している。
このような性能向上を伴うアップデートはほとんどのスクリプト言語においても同様に行われており、
スクリプト言語の性能は従来考えられているよりも相当程度に改善されている可能性がある。
本研究ではwebサービスに用いられるweb開発に用いられる主要スクリプト言語であるPython3,PHP,node.js,rubyの最新バージョンを対象にして、
実行時間に関する比較検討を行う。

本論文の構成は以下の通りである。\\
\ref{cha:script-language}章で計測に使用するスクリプト言語4種類について概要を述べる。
また、\ref{cha:environment}章では計測に使ったpcの構成とスクリプト言語のバージョンについて述べる。
\ref{cha:program}章は計測に使用したプログラムの概要を述べる。
\ref{cha:result}章ではプログラムの計測結果を表にし、その結果を考察する。
最後に、\ref{cha:conclusion} 章では、本論文のまとめと今後の課題を述べる。
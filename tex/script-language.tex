\chapter{計測対象とするスクリプト言語の概要}
\label{cha:script-language}

本章では、本研究において実行時間の比較検討の対象となる 4 種類のスクリプト言語の概要について述べる。

\section{Python}
\label{cha:script-language:python}

Python は、オープンソースのオブジェクト指向スクリプト言語であり、1991 年に最初のバージョンが公開された。
Python は、同じ処理を行うプログラムは誰が書いても同じになる事を目指して開発されたスクリプト言語であり、
インデントによってプログラムのブロックを定義すると言う特徴がある。
ほとんどのプログラミング言語においては、インデントは意味を持たず開発者によるプログラムの可読性を
高めるために任意で使用されていたものであるが、Python ではインデントを用いてブロックを表現する事によって
プログラムの構造が似たものとなるため、可読性が高いと言われている。
Python は統計や解析、分析に長けたライブラリが充実しており、研究分野でよく使われる。
特に近年では、ディープラーニングと呼ばれる機械学習手法に注目が集まっており、これを実現するための
ライブラリが充実している Python には大きな注目が集まっている。

\section{PHP}
\label{cha:script-language:php}

PHP は、オープンソースのスクリプト言語であり、1995 年に最初のバージョンが公開された。
PHP は Hypertext Preprocessor の略称であり、主に動的な Web ページを生成する目的で開発が始まった。
そのため、Web サービスや Web アプリケーション開発に関する標準ライブラリが豊富で、様々な Web サーバ上で利用されている。
例えば、PHP で書かれた Web サービスには Facebook~\cite{Facebook}, Wikipedia~\cite{Wikipedia},
Slack~\cite{Slack} などが挙げられる。
PHP の制御系は \verb|<?php ?>| で囲まれた部分を読み取って解釈し、プログラムを実行する。
ファイルの一部分に PHP のプログラムを記述できると言う性質上、マークアップ言語である HTML に埋め込んで利用される事も多い。
また、C や Perl の影響を強く受けており、文法やプログラムの構造がこれらのプログラミング言語に類似しているため、
それらのプログラミング経験者は学習が容易である。

PHP は 2015 年に新バージョンとなる PHP7~\cite{PHP7} がリリースされたが、
PHP7 は以前のバージョンである PHP5.6 と比較すると、ほとんどの互換性を維持したまま
約二倍の性能向上に成功している。さらに命令呼び出し回数の削減や検索手法の改善、メモリ使用量の削減なども
行われ、総合的なパフォーマンスが向上している。

\section{Node.js}
\label{cha:script-language:nodejs}

Node.js は非同期処理を行うアプリケーションを作成するために、これまで Web ブラウザ上で実行される事を前提としていた
JavaScript をそれ以外の場面でも実行できるようにした JavaScript 実行環境であり、2009 年に最初のバージョンが公開された。
メモリ消費量が少ないため、小規模の運用を行う場合では、他の環境と比べると総合的なパフォーマンスが高いとされる。
また、ネットワーク通信やファイルの読み書きに関する処理において、処理待ちによってブロックされることが少ない非同期方式
による処理を基本とするライブラリ設計がなされている。
Web サーバでは、サーバへの接続台数が 1 万台を超えると処理が遅くなる C10K 問題~\cite{C10k} と呼ばれる
課題に悩まされてきたが、非同期処理を基本とした Node.js を用いる事で、この問題が比較的簡単に解決される。
そのため、リアルタイム性の問われる Web サービスや Web アプリケーションの開発に長けているとされる。

\section{Ruby}
\label{cha:script-language:ruby}

Ruby は日本人によって開発されたオープンソースのオブジェクト指向プログラミング言語であり、1995 年に最初のバージョンが公開された。
Enjoy Programming! を設計思想として開発されたプログラミング言語で、プログラムの記述量が少ない、構文がシンプル、標準ライブラリが
高機能などの特徴がある。また、日本発のプログラミング言語では初めて国際標準規格に認定された。
また、Ruby on Rails~\cite{Rails} と言う Web サービスや Web アプリケーションを開発するためのフレームワークが有名であり、
このフレームワークを用いて開発された Web サービスや Web アプリケーションが世界中で数多く存在している。
さらに、日本発のプログラミング言語であるため、日本語の資料が豊富であり、日本において初学者が最も容易に学習できる
プログラミング言語の一つである。

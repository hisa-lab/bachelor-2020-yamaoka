\documentclass[11pt]{sty/oecu-thesis}
\usepackage{cite}
\usepackage[dvipdfmx]{graphicx}
\usepackage{subfigure}
\usepackage{sty/lcaption}
\usepackage{times}
\usepackage{url}
\usepackage{amsmath}

% もう少し具体的なタイトルにする。
\title[スクリプト言語のパフォーマンスについての検証]{スクリプト言語のパフォーマンスについての検証}
\author{山岡 風太}
\date{{令和}\rensuji{2}年\rensuji{12}月\rensuji{14}日}
\学生番号{HT18A102}
\指導教員{久松 潤之 准教授}

% 特別研究の場合はコメントをはずす.卒業研究の場合はコメントアウトする.
%\論文種別{特別研究論文}
\年度{令和2}

\所属{総合情報学部 情報学科}


\begin{document}

\input{tex/macro}

%\makeextratitle
\maketitle
\pagenumbering{roman}
\begin{abstract}
  近年、スマートフォン等の個人用端末の普及によってインターネット上のWEBサービスは多くの人に利用されている。 
  Web サービスを構築する際には PHP 等のスクリプト言語と呼ばれるプログラミング言語を用いるのが一般的である。
  スクリプト言語はC++ 等と比較するとコンパイルが不要という利点があるが、それによって実行時間が長くなりがちと言う問題が一般的に知られている。
  しかし、近年、需要の増大にともなって、各種スクリプト言語はバージョンアップ時に実行時間を含めたパフォーマンス問題の改善に力を入れている。
  また、Node.js など近年出現したスクリプト言語あるいはフレームワークは、初期段階からパフォーマンス問題に対して意欲的に取り組んでいる。
  そのため、これらのスクリプト言語のパフォーマンスは、従来考えられているよりも相当程度に改善されている可能性がある。
  そこで、本研究では各種スクリプト言語の最新のバージョンを対象にして、実行時間のパフォーマンスに関する検証を行う。
  
\keywords % 主な用語
  スクリプト言語\quad
  Ubuntu\quad
  Python~\cite{Python}\quad
  PHP\quad
  Ruby\quad  
  Node.js\quad

\end{abstract}



\tableofcontents
% 以下の二つは,論文のフォーマットにそっていないが,
% 確認用のためにつける.論文提出時には,コメントアウトする.
\listoffigures
\listoftables
\cleardoublepage

\setcounter{page}{1}
\pagenumbering{arabic}

\chapter{はじめに}
\label{cha:intro}
近年、Webサービスはスクリプト言語を使用することが多い。また、スクリプト言語はwebサービスの需要増加に伴って増加傾向にある。
スクリプト言語はC++などの言語に比べて実行速度が遅かったが、近年スクリプト言語開発者による最適化が進んでいる。
それによって、スクリプト言語のパフォーマンスは従来考えられていたものより改善されている可能性がある。
本研究では、4つのスクリプト言語を用意してパフォーマンスの検証を行った。
本論文の構成は以下の通りである。
まず、\ref{cha:related} 章では、本研究で参考にしたwebサイトについて記す。
その後、\ref{cha:script-language}章で計測に使用するスクリプト言語4種類について概要を述べる。
また、\ref{cha:environment}章では計測に使ったpcの構成とスクリプト言語のバージョンについて記す。
\ref{cha:program}章は計測に使用したプログラムの概要を述べる。
\ref{cha:result}章では計測結果を表にして、その結果を考察する。
最後に、\ref{cha:conclusion} 章では、本論文のまとめと今後の課題を述べる。

\chapter{関連研究}
\label{cha:related}
引用の書き方

例えば、\cite{NS2} では、~~
メモ https://benchmarksgame-team.pages.debian.net/benchmarksgame/fastest/php.html


\chapter{計測対象とするスクリプト言語の概要}
\label{cha:script-language}

本章では、本研究において実行時間の比較検討の対象となる 4 種類のスクリプト言語の概要について述べる。

\section{Python}
\label{cha:script-language:python}

Python は、オープンソースのオブジェクト指向スクリプト言語であり、1991 年に最初のバージョンが公開された。
Python は、同じ処理を行うプログラムは誰が書いても同じになる事を目指して開発されたスクリプト言語であり、
インデントによってプログラムのブロックを定義すると言う特徴がある。
ほとんどのプログラミング言語においては、インデントは意味を持たず開発者によるプログラムの可読性を
高めるために任意で使用されていたものであるが、Python ではインデントを用いてブロックを表現する事によって
プログラムの構造が似たものとなるため、可読性が高いと言われている。
Python は統計や解析、分析に長けたライブラリが充実しており、研究分野でよく使われる。
特に近年では、ディープラーニングと呼ばれる機械学習手法に注目が集まっており、これを実現するための
ライブラリが充実している Python には大きな注目が集まっている。

\section{PHP}
\label{cha:script-language:php}

PHP は、オープンソースのスクリプト言語であり、1995 年に最初のバージョンが公開された。
PHP は Hypertext Preprocessor の略称であり、主に動的な Web ページを生成する目的で開発が始まった。
そのため、Web サービスや Web アプリケーション開発に関する標準ライブラリが豊富で、様々な Web サーバ上で利用されている。
例えば、PHP で書かれた Web サービスには Facebook~\cite{Facebook}, Wikipedia~\cite{Wikipedia},
Slack~\cite{Slack} などが挙げられる。
PHP の制御系は \verb|<?php ?>| で囲まれた部分を読み取って解釈し、プログラムを実行する。
ファイルの一部分に PHP のプログラムを記述できると言う性質上、マークアップ言語である HTML に埋め込んで利用される事も多い。
また、C や Perl の影響を強く受けており、文法やプログラムの構造がこれらのプログラミング言語に類似しているため、
それらのプログラミング経験者は学習が容易である。

PHP は 2015 年に新バージョンとなる PHP7~\cite{PHP7} がリリースされたが、
PHP7 は以前のバージョンである PHP5.6 と比較すると、ほとんどの互換性を維持したまま
約二倍の性能向上に成功している。さらに命令呼び出し回数の削減や検索手法の改善、メモリ使用量の削減なども
行われ、総合的なパフォーマンスが向上している。

\section{Node.js}
\label{cha:script-language:nodejs}

Node.js は非同期処理を行うアプリケーションを作成するために、これまで Web ブラウザ上で実行される事を前提としていた
JavaScript をそれ以外の場面でも実行できるようにした JavaScript 実行環境であり、2009 年に最初のバージョンが公開された。
メモリ消費量が少ないため、小規模の運用を行う場合では、他の環境と比べると総合的なパフォーマンスが高いとされる。
また、ネットワーク通信やファイルの読み書きに関する処理において、処理待ちによってブロックされることが少ない非同期方式
による処理を基本とするライブラリ設計がなされている。
Web サーバでは、サーバへの接続台数が 1 万台を超えると処理が遅くなる C10K 問題~\cite{C10k} と呼ばれる
課題に悩まされてきたが、非同期処理を基本とした Node.js を用いる事で、この問題が比較的簡単に解決される。
そのため、リアルタイム性の問われる Web サービスや Web アプリケーションの開発に長けているとされる。

\section{Ruby}
\label{cha:script-language:ruby}

Ruby は日本人によって開発されたオープンソースのオブジェクト指向プログラミング言語であり、1995 年に最初のバージョンが公開された。
Enjoy Programming! を設計思想として開発されたプログラミング言語で、プログラムの記述量が少ない、構文がシンプル、標準ライブラリが
高機能などの特徴がある。また、日本発のプログラミング言語では初めて国際標準規格に認定された。
また、Ruby on Rails~\cite{Rails} と言う Web サービスや Web アプリケーションを開発するためのフレームワークが有名であり、
このフレームワークを用いて開発された Web サービスや Web アプリケーションが世界中で数多く存在している。
さらに、日本発のプログラミング言語であるため、日本語の資料が豊富であり、日本において初学者が最も容易に学習できる
プログラミング言語の一つである。


\chapter{計測環境}
\label{cha:environment}

本研究では、\ref{cha:program}~章で述べたプログラムを実行し、スクリプト言語ごとに実行時間の比較検討を行うための計測環境について述べる。
表~\ref{tbl:environment} に、プログラムを実行する計算機端末の概要を示す。
本研究では、この計算機端末にインストールされた Windows 上に存在する Windows Subsystem for Linux (WSL2) を用いて
Linux の仮想環境を構築し、その環境下で作成したプログラムを実行する。
Linux のディストリビューションは Ubuntu~\cite{Ubuntu} とし、バージョンは 20.04. を用いた。
また、本研究においてプログラムを実行する際に使用する各種スクリプト言語のバージョンを表~\ref{tbl:language-version} に示す。
これらは、ともに 2020 年 8 月 29 日時点の最新バージョンである。

\begin{table}[htbp]
\begin{center}
\caption{計測環境}
\label{tbl:environment}
\begin{tabular}{|l||l|} \hline
CPU & Intel Core i5-8400 2.80GHz $\times 6$ \\ \hline
Memory & 16GB \\ \hline
Storage & SSD 512GB \\ \hline
OS & Microsoft Windows 10 Home \\ \hline
\end{tabular}
\end{center}
\end{table}

\begin{table}[htbp]
\begin{center}
\caption{スクリプト言語のバージョン}
\label{tbl:language-version}
\begin{tabular}{|l|l|} \hline
スクリプト言語 & バージョン \\ \hline \hline
Python & 3.8.2 \\ \hline
PHP & 7.4.9 \\ \hline
Node.js & 12.18.3 \\ \hline
Ruby & 2.7.1 \\ \hline
\end{tabular}
\end{center}
\end{table}


\chapter{計測したプログラム}
\label{cha:program}
計測に使用したプログラムは\cite{Benchmark}を参考にして自作し計測を行った。
\section{バブルソート}
数字のソートに関する性能を評価するために、バブルソートを行うプログラムを作成した。
バブルソートはソートアルゴリズムの一つで実行の速度が遅く実用的ではないが、仕組みが理解しやすいため初学者の学習によく使われる。
このアルゴリズムは、隣合う数字の大小を比べて入れ替えをすべての要素について行う。一度目の比較が終わると、数列の最大値が確定し、右端に移動した状態になる。
次に、この動作を確定した最大値をのぞいた数列で行うことを繰り返すことでソートを行う。

Python3で実装したものを図\ref{fig:b-rb}に示す。
このプログラムはテキストファイルから数列を読み取り、ソートした結果をテキストファイルに書き込む。
図\ref{fig:b-rb}の1行目で関数bsort()を定義しており、2行目から6行目でアルゴリズムを実装している。
図\ref{fig:b-rb}のそのほかの処理はテキストファイルからソートする数字を取り出す処理である。
今回作成したプログラムの計算量は O$(n^{2})$ となる。
これはソートアルゴリズムとしては遅い。例えば、クイックソートの計算量はO$(n\log{n})$とバブルソートよりも計算量が少ない。
しかし、現在のコンピューターの性能では負荷の少ないクイックソートの計算は一瞬で終了してしまうため計測が難しいと考えた。
そのため本研究では、計算量が多く、スクリプト言語の差が顕著に出ることが予想されるバブルソートを計測することにした。

\section{フィボナッチ数列}
スクリプト言語の再起呼び出しの性能を測定するために、フィボナッチ数列を算出するプログラムを作成した。
フィボナッチ数列とは「1,1,2,3,5,8,13,21,34,55,89,144」という様に数列のそれぞれの数が1つ前と2つ前の数の和になっている。
フィボナッチ数列は漸化式\ref{eq:fibonacci} と表せる。
\begin{eqnarray} \label{eq:fibonacci}
  F_{0}&=&0 \nonumber \\
  F_{1}&=&1 \\
  F_{n+1}&=&F_{n}+F_{n+1}(n≥0)\nonumber
\end{eqnarray}
この漸化式を再起処理を使って再現し実装する。

Python3で実装したものを図\ref{fig:f-py}に示す。
図\ref{fig:f-py}の3行目に定義したfib()関数を7行目で再起呼び出しを行っている。
fib()の引数が0か1になるまで引数を1ずつ減らしながらfib()を呼び出し続ける。
再起呼び出しを使ったプログラムはforなどのループ処理に比べて、
関数呼び出し時の負担が何度も存在するため実行時間が増加する傾向にある。
同様に、今回のプログラムは導出する数が1大きくなるごとに再起呼び出しの回数が指数関数的に増加する。
今回作成したプログラムの計算量は O$((\frac{1 +\sqrt{5}}{2})^{n-1})$ となる。
このような時間のかかる漸化式を使った実装は避けられており実用的でない。
一般的には言語の機能の一つであるメモ化などを使って実装することが多い。
しかし、本研究では再起呼び出しに関する処理能力を計測することを目的としていたため、効率の悪い漸化式での実装を行った。

\section{円周率の算出}
スクリプト言語の計算能力を測定するために、円周率を求めるプログラムを作成した。
円周率を求めるためにライプニッツ級数を使用した。
ライプニッツ級数は次の級数\ref{eq:leibniz}と表せる。
\begin{eqnarray} \label{eq:leibniz}
\sum_{n=0}^{\infty}\frac{(-1)^n}{2n+1}=\frac{\pi}{4}
\end{eqnarray}
また、この式は式\ref{eq:leibniz2}と表せる。
\begin{eqnarray} \label{eq:leibniz2}
1-\frac{1}{3}+\frac{1}{5}-\frac{1}{7}+\cdots=\frac{\pi}{4}
\end{eqnarray}
級数の実行回数を増やすと円周率の精度が上がる。
10億回程度実行すると円周率が10桁まで正しく算出される。
円周率の計算方法の中では収束が遅く効率は悪いが、
現在のコンピューターでその他の方法を使うと非常に早く収束してしまい測定が難しい。
よって、ライプニッツ級数はプログラミング言語の性能を測るのに適していると考えた。

Python3で実装したものを図\ref{fig:p-py}に示す。
図\ref{fig:p-py}では6行目から9行目を使い式\ref{eq:leibniz}の左辺を再現している。
また、10行目で左辺を4倍することでπを求めている。
またPHPで実装したものを図\ref{fig:p-php}に示す。
2行目から11行目で級数の実装をしており、その他の部分は導出した解をテキストファイルに書き込み処理である。
この二つを比べると級数の実装方法にはほぼ差異がないことがわかる。
残りの2つのスクリプト言語での実装に関しても、同じような計算量になるように注意して記述を行った。
このプログラムの計算量はO(n)となり、実行回数に比例する。

\section{正規表現}
スクリプト言語の正規表現に関する処理能力を測定するためにURLを判別するプログラムを作成した。
任意の行数のurlが混ざったテキストファイルを開き、そのデータから正規表現を使ってURLを取り出し出力する。
URLの判別には正規表現を使う。行の数を増やしていき、実行の速度を測る。

Python3で実装したものを図\ref{fig:s-py}に示す。
図\ref{fig:s-py}では正規表現を扱うライブラリ"re"をインポートして関数re.findallを使用していることがわかる。
また、PHPで書かれたプログラム図\ref{fig:s-php}ではpreg-match関数が使われている。
これらの関数は結果としては同じふるまいをする関数だが、言語ごとに実行速度が異なる結果になることが予想できた。

\begin{lstlisting}[label={prg:gst}, caption={GStreamer コマンドライン}, basicstyle=\ttfamily\footnotesize, frame=single]
gst-launch-1.0 -v rtmpsrc location=rtmp://localhost/live/live \
! flvdemux ! h264parse ! decodebin ! videoscale add-borders=true \
! video/x-raw,width=888,height=480 \
! x264enc bitrate=900 speed-preset=1 tune=zerolatency \
! rtph264pay config-interval=-1 pt={pt} \
! udpsink host=127.0.0.1 port={port}
\end{lstlisting}

\begin{figure}[tb]
    \centering
    \includegraphics[width=13.5cm,keepaspectratio]{figure/b-rb.PNG}
    \caption{Python3 バブルソート}
    \label{fig:b-rb}
\end{figure}

\begin{figure}[tb]
    \centering
    \includegraphics[width=13.5cm,keepaspectratio]{figure/f-py.PNG}
    \caption{Python3 フィボナッチ数列}
    \label{fig:f-py}
\end{figure}

\begin{figure}[tb]
    \centering
    \includegraphics[width=13.5cm,keepaspectratio]{figure/p-py.PNG}
    \caption{Python3 円周率の算出}
    \label{fig:p-py}
\end{figure}

\begin{figure}[tb]
    \centering
    \includegraphics[width=13.5cm,keepaspectratio]{figure/p-php.PNG}
    \caption{PHP 円周率の算出}
    \label{fig:p-php}
\end{figure}

\clearpage
\begin{figure}[tb]
    \centering
        \includegraphics[width=13.5cm,keepaspectratio]{figure/s-py.PNG}
        \caption{Python3 正規表現}
        \label{fig:s-py}
\end{figure}

\begin{figure}[tb]
    \centering
        \includegraphics[width=13.5cm,keepaspectratio]{figure/s-php.PNG}
        \caption{PHP 正規表現}
        \label{fig:s-php}
\end{figure}
\chapter{計測結果}
\label{cha:result}
計測結果を書く
\section{フィボナッチ数列}
図\ref{fig:f-average}
\insertfigpng{f-average}{フィボナッチ数列 平均}
図\ref{fig:f-dispersion}
\insertfigpng{f-dispersion}{フィボナッチ数列 分散}

\section{円周率の算出}
図\ref{fig:p-average}
\insertfigpng{p-average}{円周率の算出 平均}
図\ref{fig:p-dispersion}
\insertfigpng{p-dispersion}{円周率の算出 分散}


\section{正規表現}
図\ref{fig:s-average}
\insertfigpng{s-average}{正規表現 平均}
図\ref{fig:s-dispersion}
\insertfigpng{s-dispersion}{正規表現 分散}

\chapter{まとめと今後の課題}
\label{cha:conclusion}

 本研究では、主要な 4 種類のスクリプト言語である Python, PHP, Ruby, Node.js のパフォーマンスの検証を実行速度の観点から行った。
 また、その結果を表にして示しその考察を行った。

 今後の課題としては、パフォーマンス検証の評価を実行速度だけではなく、CPUの使用率やメモリー使用率などを加味して行うべきだろう。
\input{tex/acknowledgements}

\bibliographystyle{junsrt}
\bibliography{bib/myrefs}


\end{document}

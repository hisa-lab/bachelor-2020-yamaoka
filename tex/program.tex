\chapter{計測したプログラム}
\label{cha:program}
計測したプログラム3種類の概要を書く
\section{フィボナッチ数列}
再帰処理 指数関数時間
\section{円周率の算出}
ライプニッツ級数 導出法の中では効率は悪い 級数の実行回数を増やすと円周率の精度が上がる
\section{正規表現}
100行のurlが混ざったテキストファイルを開き、そのデータからURLを取り出して出力する。
URLの判別には正規表現を使う。
一連の流れを連続で行う回数を増やしていき、実行の速度をはかる。
各20回ずつ計測して平均と分散を調べる。
